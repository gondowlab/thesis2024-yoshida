\section{おわりに}
本研究では、記号実行とファジングの解析手法を組み合わせることで、スケーラビリティと精度の両立を目指すSpectre Gadgetの検出手法を提案した。記号実行とファジングにはそれぞれ利点と欠点が存在するが、両者を組み合わせることで、それぞれの欠点を補完し合い、スケーラビリティと精度を両立させることを目的とした。
記号実行フェーズでは、ネストされた分岐予測ミスの回数を制限することで、投機的状態の探索範囲を縮小し、記号実行のスケーラビリティを向上させる手法を用いた。また、ファジングフェーズでは記号実行で得られた解析結果を活用し、記号実行で探索されなかった状態を効率的に探索することを試みた。予備実験では、いくつかの検体において記号実行のスケーラビリティが大幅に向上することを確認した。さらに、提案手法全体としては、既存手法よりも効率的にSpectre Gadgetを検出した検体を確認した。
今後の展望として、スケーラビリティと精度の向上を図るため、最適なOrder値の探索やスコアリング手法の改善が必要であると考える。これにより、提案手法がより多くの検体に対して有効に機能するようになると考える。

