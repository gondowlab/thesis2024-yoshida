\section{今後の展望}

\subsection{最適なOrder値の模索}
今回の提案手法の評価では、Order値を1に設定して実験を行った。この設定により、記号実行フェーズで探索される投機的状態数は減少したが、その分ファジングフェーズで探索するべき投機的状態数が増加した。ファジングのターゲット状態が増加すると、ファジング特有のランダム性の影響で、検体にType2やType3のGadgetが多く含まれる場合にfalse negativeが増加する可能性がある。
一方で、Order値を上げて記号実行フェーズで探索する投機的状態数を増やすと、記号実行のスケーラビリティが低下する問題が生じる。よって今後の研究では、このトレードオフを考慮して、より多くの検体に対して様々なOrder値を設定し詳細な実験を行うことで、スケーラビリティと精度の両立を実現する最適なOrder値を特定する必要があると考えられる。\par

\subsection{スコアリング手法の改善}
提案手法では、テストケースが通過したTarget Directionの種類数が多いほどスコアが高くなるという単純なスコアリング手法を採用している。しかし、予備実験の結果から、この手法では十分に効果的にテストケースをターゲット状態に誘導できていないことが示唆された。これを改善するためには、より詳細な情報を活用してテストケースを評価する手法が必要であると考える。例えば、分岐方向の通過順序を考慮したり、ターゲット状態ごとにスコアを個別に計測することで、テストケースの有効性をより正確に評価できる可能性がある。また、一般的なdirected fuzzer\cite{bohme2017directed}のアプローチを参考にし、制御フローグラフ上でターゲット状態までの距離を考慮する手法も有効と考えられる。ただし、提案手法では、ターゲット状態は投機的な状態であるため、通常の制御フローグラフ上には存在しない。したがって、投機的なパスを含めた制御フローグラフを構築する必要があると考えられる。いずれにしても、スコアリング手法が複雑化すると、それに伴いファジングのスループットが低下する可能性があるため、効率的な設計が必要となる。

\subsection{テストドライバの設計}
記号実行やファジングを用いてプログラムを適切にテストするには、テストドライバが必要不可欠である。テストドライバでは、記号実行において記号化する変数を指定したり、ファジングにおいて生成されたシードを用いてプログラムの入力を正しく初期化する役割を担う。また、これらの変数を対象の関数などに渡して実行する。テストドライバの設計は解析の精度やスケーラビリティに大きく影響する。
例えば、記号実行の場合、適切に変数を記号化しなければ特定のパスを探索できない可能性がある。一方で、すべての変数を記号化すると状態爆発が発生し、スケーラビリティが著しく低下するリスクがある。同様に、ファジングでは、変数が正しく初期化されていない場合、早期にエラーが発生してカバレッジが向上しない可能性がある。
本研究の実験では、すべての検体に対して自作のテストドライバを使用した。そのため、解析の精度やスケーラビリティがこれらのドライバの設計に依存していた可能性がある。記号実行においては、制御フローに影響するすべての変数を特定し記号化することで、全てのパスを探索できると考える。ファジングではFudge\cite{babic2019fudge}のようなファジングドライバを自動生成するツールを活用することで、適切なドライバを用いた実験が可能となると考える。

\subsection{実世界のプログラムにおけるGadgetの調査}
本研究の実験では、実世界のプログラムにおけるground truthが限られているため、他のツールとの定量的な比較が困難であった。この課題に対処するため、既知のGadgetを埋め込んだデータセットを作成し、実験を行った。具体的には、提案手法の有効性を評価するため、検体全体にわたってType1のGadgetとそれ以外のGadgetを均等に10個ずつ埋め込んだ。
しかし、実世界のプログラムにおいて各TypeのGadgetがどのような割合で存在しているかは明らかではない。先行研究\cite{oleksenko2020specfuzz}の調査では、一部の検体ではType1のGadgetが最も多く含まれていることが示されているが、この割合によっては適切なOrder値が変化する可能性があるためより広範囲な調査が必要である。そのため、多くの実世界のプログラムを対象に各TypeのGadgetの割合を調査することが、今後の研究において重要であると考えられる。

